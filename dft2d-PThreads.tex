\documentclass[10pt]{article}
\usepackage{times,epsfig}
\setlength{\textwidth}{6.5in}
\setlength{\textheight}{9in}
\setlength{\oddsidemargin}{0.0in}
\setlength{\evensidemargin}{0.0in}
\setlength{\topmargin}{-0.5in}
\usepackage{amsmath}
\input{../teach}
\input{../TAName}
\begin{document}
\header{Project 2 -- The Threaded Two--Dimensional Discrete Fourier Transform}
{Assigned: Sept 21, 2011}
{Due: Oct 3, 2011, 11:59pm}
{ECE4893/8893}
{Fall Semester, 2011}

\def\td{two--dimensional}
\def\od{one--dimensional}
Given a \od\ array of complex or real input values of length $N$,
the Discrete Fourier Transform consists af an array of size $N$ computed
as follows:

\begin{equation}
\label{Eq:DFT}
H[n] = \sum_{k=0}^{N-1} W^{nk}h[k]
\hspace{0.5in}\mbox{where}\ \ 
W = e^{-j2\pi/N} = \mbox{cos}(2\pi/N)-j\mbox{sin}(2\pi/N)
\hspace{0.5in}\mbox{where}\ \ 
j = \sqrt{-1}
\end{equation}
For all equations in this document, we use the following notational
conventions.
$h$ is the discrete--time sampled signal array.
$H$ is the Fourier transform array of $h$. 
$N$ is the length of the sample array, and is always
assumed to be an even power of 2.
$n$ is an index into the $h$ and $H$ arrays, and is always in the
range $0 \ldots (N-1)$.
$k$ is also an index into $h$ and $H$, and is the summation
variable when needed.
$j$ is the square root of negative one.

An important special case of the above equation is the case of a
sample of length 1.  Since the summation variable $k$ is exactly
$0$, the $W^{nk}$ term becomes $1$, and thus $H[0] = h[0]$.  The
Fourier transform of a sample set of length 1 is just the original
sample set unmodified.

As in project 1, we are going to compute a \td\ Discrete Fourier
Transform of a given input image.  As before, we first compute the
\od\ transforms on eacy row, followed by the \od\ transforms on each
column.  However, there are two major differences in this assignment
and the prior one.

\begin{enumerate}
\item We will use the much more efficient {\em Danielson--Lanczos}
approach for the \od\ transforms (described in detail below).
This approach has a running time proportional to $N log_2 N$ as
opposed to the $N^2$ running time of the prior algorithm.
\item We are going to use 16 {\em threads} on a single computing
platform to cooperate to perform the \td\ transform.  All threads will
have access to the same memory locations (ie. the original 2d
image).
\end{enumerate}
As before, each thread will do the \od\ transform on a set of rows,
and then do the \od\ transform on a set of columns.  You will
need a barrier to be sure that no thread starts the column-wise
transform before all threads have completed the row-wise transform.

{\bf Graduate Students}.  Grad students must implement their own
barrier code.  Undergrads can use the built-in {\em PThreads}
barrier routines.

\begin{Large}
\begin{center}
{\bf Description of the Danielson--Lanczos Algorithm}
\end{center}
\end{Large}
From equation~\ref{Eq:DFT}, it appears that to compute the DFT for 
a sample of length $N$, it must take $N^2$ operations, since to compute
each element of $H$ requires a summation overall all samples in $h$.
However, Danielson and Lanczos demonstrated a method that reduces the
number of operations from $N^2$ to $N\mbox{log}_2(N)$.  The insight
of Danielson and Lanczos was that the FFT of an array of length $N$
is in fact the sum of two smaller FFT's of length $N/2$, where the
first half--length FFT contains only the even numbered samples, and the
second contains only the odd numbered samples.   The proof of the
{\em Danielson--Lanczos Lemma} is quite simple, as follows:
\begin{equation}
\label{Eq:Danielson}
\begin{aligned}
H[n] & = \sum_{k=0}^{N-1} e^{-j2\pi kn/N} h[k] \\
& = \sum_{k=0}^{N/2-1} e^{-j2\pi 2kn/N} h[2k] +
    \sum_{k=0}^{N/2-1} e^{-j2\pi (2k+1)n/N} h[2k+1] \\
& = \sum_{k=0}^{N/2-1} e^{-j2\pi kn/(N/2)} h[2k] +
    \sum_{k=0}^{N/2-1} e^{-j2\pi kn/(N/2)} e^{-j2\pi n/N} h[2k+1] \\
& = \sum_{k=0}^{N/2-1} e^{-j2\pi kn/(N/2)} h[2k] +
    W^n \sum_{k=0}^{N/2-1} e^{-j2\pi kn/(N/2)} h[2k+1] \\
H[n] & = H_{n\ \mbox{\tiny mod\ (N/2)}}^e + W^n H_{n\ \mbox{\tiny mod\ (N/2)}}^o
\end{aligned}
\end{equation}
where $H_n^e$ refers to the $n^{th}$ element of the Fourier transform
of length $N/2$ formed from the even numbered elements of the original
length $N$ samples, and $H_n^o$ refers to the $n^{th}$ element of the
odd numbered samples.  Thus the Fourier transform of an array 
of length $N$ can be computed by summing two transforms of length
$N/2$.  Then each of the $N/2$ transforms can be computed as the sum
of two transforms of length $N/4$, which in turn can be computed as
the sum of two transforms of length $N/8$.  This process can be repeated
until we have a transform of length $1$, which we already know can
trivially be computed.

Using the equation for the Fourier transform and the Danielson--Lanczos
lemma, we can design an easy--to--implement and efficient algorithm for 
computing the Discrete Fourier Transform for a set of signal
samples of length $N = 2^m$.  This algorithm is known as the Cooley-Tukey
algorithm.  By using the Cooley-Tukey algorithm, we can reduce the
computational complexity of the DFT computation from $N^2$ operations
to $N\mbox{log}_2(N)$

\paragraph{Step 1. Reordering the initial $h$ array.}

Consider a simple case of a sample set of length $8$, $h[0], h[1] \ldots h[7]$.
Dividing this into the even samples and odd samples, we get:

\begin{equation}
\begin{aligned}
H^e & = h[0] + h[2] + h[4] + h[6] \\
H^o & = h[1] + h[3] + h[5] + h[7]
\end{aligned}
\end{equation}
Further dividing the even set into it's even and odd components, and 
similarly dividing the odd set into it's even and odd components, we get:

\begin{equation}
\label{Eq:TwoLength}
\begin{aligned}
H^{ee} & = h[0] + h[4] \\
H^{eo} & = h[2] + h[6] \\
H^{oe} & = h[1] + h[5] \\
H^{oo} & = h[3] + h[7]
\end{aligned}
\end{equation}
Finally, dividing each of these into it's even and odd components we get:

\begin{equation}
\label{Eq:OneLength}
\begin{aligned}
H^{eee} & = h[0] \\
H^{eeo} & = h[4] \\
H^{eoe} & = h[2] \\
H^{eoo} & = h[6] \\
H^{oee} & = h[1] \\
H^{oeo} & = h[5] \\
H^{ooe} & = h[3] \\
H^{ooo} & = h[7]
\end{aligned}
\end{equation}
Since each line of equation~\ref{Eq:OneLength} is just a Fourier
transform of length one, we can trivially compute each of these.
The question now becomes is there an easy way to determine which of
the $h[n]$ values corresponds to each of the $H^{xxx}$ components.
In other words, in equation~\ref{Eq:OneLength} we determined that
$H^{eoo}$ (for example) is equal to  $h[6]$, but is there a general
way to find this mapping?  It turns out that there is, by simply
assigning a $0$ value to each $e$ (even iteration) and and $1$
values to each $o$ (odd iteration), and then interpreting the
resulting binary value {\em in reverse order}.  In our example
of $H^{eoo}$, the binary value is $011$, which is the value $6$
when read from right to left.  The first step of the Cooley--Tukey
algorithm is to simply transform the original $h$ array from natural
ordering ($h[0], h[1], h[2], \ldots h[N-1]$) to the bit reversed ordering.
For an original vector of length $8$, the reverse ordering is the 
sequence shown in equation~\ref{Eq:OneLength}.  For original array
lengths other than $8$, the reversed orderings are of course different,
but always easy to compute.

After reordering the original $h$ array, we can easily compute the four
sets of Fourier transforms of length 2, since the required elements
are adjacent in the reordered array.  Referring to equation~\ref{Eq:TwoLength},
we see that the {\em ee} values are $h[0]$ and $h[4]$, which are
the first two elements in the reordered array; the {\em eo} values
are $h[2]$ and $h[6]$ which are the next two, and so on.  Similarly
we can compute the two sets of Fourier transforms of length 4, since
the {\em e} elements are the first four and the {\em o} elements 
are the next four.  Thus the bookkeeping needed for determining which
elements of the $h$ array are needed at each step is quite simple.

\paragraph{Step 2.  Precomputing the $W^n$ values.}

Recall the definition of the complex {\em Weight} factor from
equation~\ref{Eq:DFT} is:
\begin{equation}
\label{Eq:W}
\begin{aligned}
W   & = e^{-j2\pi/N} = \mbox{cos}(2\pi/N)-j\mbox{sin}(2\pi/N) \\
W^n & = e^{-j2\pi n/N} = \mbox{cos}(2\pi n/N)-j\mbox{sin}(2\pi n/N)
\end{aligned}
\end{equation}
Further recall that we need the value $W^n$ in equation~\ref{Eq:Danielson}
to combine the {\em even} and {\em odd} sub--transforms.  Since $n$
is the array index in the original $h$ array (of length $N$), there are
exactly $N$ distinct $W^n$ values ($W^0, W^1, \ldots W^{N-1}$).  Since these
are somewhat expensive to compute (two trigonometric functions), we can
save some time by precomputing the $N$ distinct weights. As it turns
out, we in fact only need to compute the first half of these weights.
For any $W^n$, we can show that:
\begin{equation}
W^{n + N/2} = -W^n
\end{equation}
The proof of this identity is trivial, and is left as an exercise for
the reader.  Using this identity, we just need to compute $W^n$ for
$n = [0, 1, \ldots (N/2-1)]$.

\paragraph{Step 3.  Do the transformation.}
Once the input array is re--ordered in the bit reversed order and the
$W$ values are computed, we can easily perform the transform by
starting with a two-sample transform of side--by--side elements
in the shuffled array, then computing a four--sample transform with
four adjacent elements, continuing until we have an $N$ element
transform.  The problem is complicated by the fact that we want
an {\em in--place} transformation.  We {\em do not} have a separate
$H$ array to store the transformed data; rather the original $h$
array is over--written with the computed $H$ elements.
 
In our example, we would first do a two-sample transform for each of the
four sets of two points:
 \{H[0], H[1]\}, \{H[2], H[3]\}, \{H[4], H[5]\}, \{H[6], H[7]\}, 

Note that above we refer to the elements with the symbol $H$ rather than
$h$, since each of the elements are the result of a one--point transform
which we already know is simply the point unmodified.
For the first two point transform, from equation~\ref{Eq:Danielson} we can
see that the first set would be:

\begin{equation}
\begin{aligned}
H[0] & = H_{0\ \mbox{\tiny mod\ (2/2)}}^e + W_2^0 H_{0\ \mbox{\tiny mod\ (2/2)}}^o 
     & = H[0] + W_2^0H[1] \\
H[1] & = H_{1\ \mbox{\tiny mod\ (2/2)}}^e + W_2^1 H_{1\ \mbox{\tiny mod\ (2/2)}}^o 
     & = H[0] + W_2^1H[1]
\end{aligned}
\end{equation}
although we have to very careful about the $W$ values above.
To clarify these weights, we use  a new notational convention
$W_N^k$.  The subscript $N$ indicates the number of points in
the transform and the superscript is of course the power.
Recall that
we pre--computed the weights in step 2 above, but these precomputed
weights were for a eight point transform (in our example).
In this step we are doing a 
two point transform, which apparently  will result in different
weight factors (since the definition of $W$ in equation~\ref{Eq:DFT}
has $N$ in the definition).
Given this, it appears that
we have to re--compute the weights for each transform size $N = 2, 4, ..$.
Luckily, this is not the case.  If we start with weights computed
for an $N$ point transform
we can prove that for a $x$ point transform (for all $x$ where
$x^m = N$ for some $m > 0$),
\begin{equation}
\label{Eq:WScaling}
W_x^k = W_N^{kN/x}        
\end{equation}
The proof of this is left as
an exercise. Returning to our example, since we have pre--computed
weights for $N=8$, we use:
\begin{equation}
\begin{aligned}
 W_2^0 & = W_{8}^{0(8/2)} = W_{8}^0\\
 W_2^1 & = W_{8}^{1(8/2)} = W_{8}^4 = -W_{8}^0
\end{aligned}
\end{equation}
Therefore, assuming we stored our pre--computed weights in array $W$, 
for the 2--point transform we end up with:

\begin{equation}
\begin{aligned}
H[0] & = H_{0\ \mbox{\tiny mod\ (2/2)}}^e + W_2^0 H_{0\ \mbox{\tiny mod\ (2/2)}}^o 
     & = H[0] + W_2^0H[1] & = H[0] + W[0]H[1] \\
H[1] & = H_{1\ \mbox{\tiny mod\ (2/2)}}^e + W_2^1 H_{1\ \mbox{\tiny mod\ (2/2)}}^o 
     & = H[0] + W_2^1H[1] & = H[0] - W[0]H[1]
\end{aligned}
\end{equation}
We would have similar equations for the other three sets of two--point
transforms.  Continuing to the four--point transforms, we end up with
(for the first set of four for example):

\begin{equation}
\begin{aligned}
H[0] & = H_{0\ \mbox{\tiny mod\ (4/2)}}^e + W_4^0 H_{0\ \mbox{\tiny mod\ (4/2)}}^o 
     & = H[0] + W_8^0H[2] & = H[0] + W[0]H[2]\\
H[1] & = H_{1\ \mbox{\tiny mod\ (4/2)}}^e + W_4^1 H_{1\ \mbox{\tiny mod\ (4/2)}}^o 
     & = H[1] + W_8^2H[3] & = H[1] + W[2]H[3]\\
H[2] & = H_{2\ \mbox{\tiny mod\ (4/2)}}^e + W_4^2 H_{2\ \mbox{\tiny mod\ (4/2)}}^o
     & = H[0] + W_8^4H[2] & = H[0] - W[0]H[2]\\
H[3] & = H_{3\ \mbox{\tiny mod\ (4/2)}}^e + W_4^3 H_{3\ \mbox{\tiny mod\ (4/2)}}^o
     & = H[1] + W_8^6H[3]&  = H[1] - W[2]H[3]
\end{aligned}
\end{equation}
In coding this, we must be careful in that we are using a 
{\em transform--in--place}, meaning the output array $H$ is in
fact the input array $h$.  Notice that in the above equations, 
we overwrite $H[0]$ in the first equation, but use it on the
right--hand--side in later equations.  This means we likely would
need one or more temporary variables store the original values
inside of this processing loop.

We continue transforming larger and larger sample sets (increasing by
a factor of two each time) until our final transform is all $N$
samples and the problem is solved.


\paragraph{Copying the Project Skeletons}
\begin{enumerate}
\item Log into {\tt jinx-login.cc} using {\tt ssh} and your prism log-in name.
\item Copy the files from the ECE8893 user account using the following
command:
\begin{verbatim}
/usr/bin/rsync -avu /nethome/ECE8893/ThreadsTransform2D .
\end{verbatim}
Be sure to notice the period at the end of the above command.
\item Change your working directory to {\tt ThreadsTransform2D}
\begin{verbatim}
cd ThreadsTransform2D
\end{verbatim}
\item Copy the provided {\tt threadDFT2d-skeleton.cc} to {\tt threadDFT2d.cc} as follows:
\begin{verbatim}
cp threadDFT2d-skeleton.cc threadDFT2d.cc
\end{verbatim}
\item Then edit {\tt threadDFT2d.cc} to implement the transform.
\begin{enumerate}
\item Create 16 threads to work in parallel to compute the \td\ DFT.
\item Implement and call the \od\ DFT for the necessary rows
using the Danielson--Lanczos approach
in the equations above
\item Use a barrier to insure all threads have completed the rows.
\item Use the same algorithm to compute the \od\ DFT for the
necessary columns.
\item The main program should wait (using a condition variable) until
all threads are done, then save the results in file
{\tt Tower-DFT2D.txt} using the {\tt SaveImageData} method in
class {\tt InputImage}.
\end{enumerate}
\item Compile your code using {\tt make} as follows:
\begin{verbatim}
make
\end{verbatim}
\item Once you have gotten the threadDFT2d program compiled and ready to test,
edit the {\tt runDFT.sh} script to change {\tt YourLastNameHere}
to give your last name.  This is how your jobs are identified to the
job scheduler.
\item Test your solution
with the provided inputs.  Testing is done by submitting your job to
the scheduler using {\tt qsub}.
\begin{verbatim}
qsub runDFT.sh
\end{verbatim}
\item You can check the status of your submitted jobs by using {\tt qstat}
\begin{verbatim}
qstat runDFT.sh
\end{verbatim}
\item When your submitted job completes, you will have two files created
in your workign directory with the outputs from the job and any errors produced by your job (hopefully
empty).
\end{enumerate}

\paragraph{Resources}
\begin{enumerate}
\item {\tt threadDFT2d-skeleton.cc} is a starting point for your program.
\item {\tt runDFT.sh} is the job submission script to use.  Be sure to change
the {\tt -N} parameter to your name.
\item {\tt Complex.cc} and {\tt Complex.h} provide a completed C++
object containing a complex (real and imaginary parts) value.
\item {\tt Makefile} is a file used by the {\tt make} command to
build threadDFT2d.
\item {\tt Tower.txt} is the input dataset, a 1024 by 1024 image of the
Tech tower in black and white.
\item {\tt after1d-correct.txt} is the expected value of the DFT
after the initial \od\ transform on each row, but before the column
transforms have been done.
\item {\tt after2d-correct.txt} is the expected output dataset,
a 1024 by 1024 matrix of the transformed values.
\item {\tt InputImage.cc} and {\tt InputImage.h} that will ease the reading
of the input data.  This object has several useful functions to help with
managing the input data.
\begin{enumerate}
\item The {\tt InputImage} constructor, which has a {\tt char*} argument
specifying the file name of the input image.
\item The {\tt GetWidth()} function that returns the width of the image.
\item The {\tt GetHeight()} function that returns the height of the image.
\item The {\tt GetImageData()} function returns a one-dimensional array
of type {\tt Complex} representing the original time-domain input values.
\item The {\tt SaveImageData} function writes a file containing the
transformed data.
\end{enumerate}
\end{enumerate}

\paragraph{Turning in your Project.}
Your assignment will be turned in automatically on the due date.
Be sure to put your code in subdirectory {\tt ThreadsTransform2D}
in your home directory on {\tt jinx-login.cc}

\end{document}
